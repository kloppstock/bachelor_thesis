\documentclass{article}

\usepackage{tikz}
\usetikzlibrary{arrows, shapes}

% make page small
\usepackage[active, tightpage]{preview}
\PreviewEnvironment{tikzpicture}
\setlength\PreviewBorder{5pt}

% define block styles
\tikzstyle{startstop} = [rectangle, draw, fill=red!25, rounded corners, minimum width=5em, text width=7em, minimum height=0.5cm, text centered]
\tikzstyle{decision} = [diamond, draw, fill=green!25, minimum width=5em, minimum height=0.5cm, text width=5em, text centered, inner sep=0pt]
\tikzstyle{process} = [rectangle, draw, fill=orange!25, minimum width=5em, minimum height=0.5cm, text width=7em, text centered]
\tikzstyle{process} = [rectangle, draw, fill=orange!25, minimum width=5em, minimum height=0.5cm, text width=7em, text centered]
\tikzstyle{subroutine} = [rectangle, draw, fill=blue!25, minimum width=5em, minimum height=0.5cm, text width=7em, text centered]
\tikzstyle{arrow} = [very thick, ->, >=stealth]

\begin{document}
\pagestyle{empty}
\begin{tikzpicture}[node distance=1.5cm]
	\node [startstop] (start) {Start};
	\node [process, below of=start] (fft) {Anwenden von FFTs auf Gradienten};
	\node [process, below of=fft] (int) {Integration im Frequenzraum};
	\node [process, below of=int] (ifft) {Rücktransformation mittels inverser FFTs};
	\node [startstop, below of=ifft] (stop) {Ende};
	\draw [arrow] (start) -- (fft);
	\draw [arrow] (fft) -- (int);
	\draw [arrow] (int) -- (ifft);
	\draw [arrow] (ifft) -- (stop);
\end{tikzpicture}
\end{document}