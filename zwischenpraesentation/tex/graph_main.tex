\documentclass{article}

\usepackage{tikz}
\usetikzlibrary{arrows, shapes}

% make page small
\usepackage[active, tightpage]{preview}
\PreviewEnvironment{tikzpicture}
\setlength\PreviewBorder{5pt}

% define block styles
\tikzstyle{startstop} = [rectangle, draw, fill=red!25, rounded corners, minimum width=5em, text width=7em, minimum height=0.5cm, text centered, node distance=6em]
\tikzstyle{decision} = [diamond, draw, fill=green!25, minimum width=5em, minimum height=0.5cm, text width=5em, text centered, node distance=6em, inner sep=0pt]
\tikzstyle{process} = [rectangle, draw, fill=orange!25, minimum width=5em, minimum height=0.5cm, text width=7em, text centered, node distance=6em]
\tikzstyle{subroutine} = [rectangle, draw, fill=blue!25, minimum width=5em, minimum height=0.5cm, text width=7em, text centered, node distance=6em]
\tikzstyle{arrow} = [very thick, ->, >=stealth]

\begin{document}
\pagestyle{empty}
\begin{tikzpicture}
	\node [startstop] (start) {Start};
	\node [process, below of=start] (param) {Parameterinitialisierung};
	\node [decision, below of=param] (alle_bilder) {Für alle Bilder};
	\node [process, below of=alle_bilder] (correction) {Korrektur};
	
	% subroutine hack
	\node [subroutine, below of=correction] (speckle) {Speckle-Tracking};
	\node [rectangle, draw, fill=blue!25, minimum width=8.5em, minimum height=2.77em] at (speckle.center) {};
	\node [subroutine] at (speckle.center) {Speckle-Tracking};
	
	% subroutine hack
	\node [subroutine, below of=speckle] (wf_rec) {Wellenfront-rekonstruktion};
	\node [rectangle, draw, fill=blue!25, minimum width=8.5em, minimum height=2.6em] at (wf_rec.center) {};
	\node [subroutine] at (wf_rec.center) {Wellenfront-rekonstruktion};
	
	\node [startstop, below of=wf_rec] (stop) {Ende};
	\draw [arrow] (start) -- (param);
	\draw [arrow] (param) -- (alle_bilder);
	\draw [arrow] (alle_bilder) -- node[anchor=east] {noch nicht fertig} (correction);
	\draw [arrow] (correction) -- (speckle);
	\draw [arrow] (speckle) -- (wf_rec);
	\draw [arrow] (alle_bilder) -- ++(3, 0) |- node[anchor=north] {fertig} (stop);
	\draw [arrow] (wf_rec) -| ++(-3.5, 6.3) -- (alle_bilder);
\end{tikzpicture}
\end{document}