\documentclass{article}

\usepackage{times}
\usepackage[T1]{fontenc}

\usepackage{tikz}
\usetikzlibrary{arrows, shapes, positioning, shapes.multipart}

% make page small
\usepackage[active, tightpage]{preview}
\PreviewEnvironment{tikzpicture}
\setlength\PreviewBorder{5pt}

% define box for subroutine
\makeatletter
\pgfdeclareshape{subroutinebox}{
	\inheritsavedanchors[from=rectangle]%
	\inheritanchorborder[from=rectangle]%
	\inheritanchor[from=rectangle]{center}%
	  \foreach \anchor in {north,north west,north east,center,west,east,mid,
		mid west,mid east,base,base west,base east,south,south west,south east}{%
		\inheritanchor[from=rectangle]{\anchor}}%
	\inheritbackgroundpath[from=rectangle]
	\foregroundpath{
		\northeast
		\pgf@xa=\pgf@x
		\pgf@ya=\pgf@y
		\southwest
		\pgf@xb=\pgf@x
		\pgf@yb=\pgf@y
		\pgfpathmoveto{\pgfpoint{\pgf@xa-4pt}{\pgf@ya}}
		\pgfpathlineto{\pgfpoint{\pgf@xa-4pt}{\pgf@yb}}
		\pgfpathmoveto{\pgfpoint{\pgf@xb+4pt}{\pgf@ya}}
		\pgfpathlineto{\pgfpoint{\pgf@xb+4pt}{\pgf@yb}}
	}
}

% define block styles
\tikzstyle{startstop} = [rectangle, draw, fill=red!25, rounded corners, minimum width=5em, text width=7em, minimum height=0.5cm, text centered]
\tikzstyle{decision} = [diamond, draw, fill=green!25, minimum width=5em, minimum height=0.5cm, text width=5em, text centered, inner sep=0pt]
\tikzstyle{process} = [rectangle, draw, fill=orange!25, minimum width=5em, minimum height=0.5cm, text width=7em, text centered]
\tikzstyle{subroutine} = [subroutinebox, draw, fill=blue!25, minimum width=5em, minimum height=0.5cm, text width=7em, text centered]
\tikzstyle{arrow} = [very thick, ->, >=stealth]

\begin{document}
\pagestyle{empty}
\begin{tikzpicture}[node distance=4em]
	\node [startstop] (start) {Start};
	\node [process, below of=start] (param) {Parameterinitialisierung};
	\node [decision, below of=param, node distance=7em] (alle_bilder) {Für alle Bildpaare $N$};
	\node [process, below of=alle_bilder, yshift=-2em] (correction) {Korrektur};
	\node [subroutine, below of=correction] (speckle) {Speckle-Tracking};
	\node [subroutine, below of=speckle] (wf_rec) {Gradientenintegration};
	\node [startstop, below of=wf_rec] (stop) {Ende};
	
	\draw [arrow] (start) -- (param);
	\draw [arrow] (param) -- (alle_bilder);
	\draw [arrow] (alle_bilder) -- node[anchor=east] {weitere vorhanden} (correction);
	\draw [arrow] (correction) -- (speckle);
	\draw [arrow] (speckle) -- (wf_rec);
	\draw [arrow] (alle_bilder) -- ++(3, 0) |- node[anchor=north] {alle bearbeitet} (stop);
	\draw [arrow] (wf_rec) -- ++(-3.5, 0) |- (alle_bilder);
\end{tikzpicture}
\end{document}