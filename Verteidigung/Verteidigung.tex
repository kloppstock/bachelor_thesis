\documentclass[10pt]{beamer}

\usepackage[ngerman]{babel}
\usepackage{amsmath}
\usepackage{bbm}
\usepackage{tikz}

\usepackage{tabularx}
\usepackage{graphicx}
\usepackage{subcaption}
\usepackage{url}
\usepackage{eurosym}
\usepackage{listings}
\usepackage{datetime}
\usepackage[style=authoryear, backend=biber]{biblatex}

\usepackage{epstopdf}

\usepackage{multirow}
\usepackage{colortbl}
\usepackage{booktabs}

\input{theme/theme}

% define blocks for flowchart
\usetikzlibrary{shapes.geometric, arrows}
\date{\today}
\institute[TI TUD]{Institut für Technische Informatik, Professur für Rechnerarchitektur -- TU Dresden}
\title[PWFRA]{Parallelisierung des Wellenfrontrekonstruktionsalgorithmus auf Multicore-Prozessoren}
%\subtitle{Zwischenpräsentation}
\author[Schenke]{Jonas~Schenke}

\DeclareCiteCommand{\citejournal}
{\usebibmacro{prenote}}
{\usebibmacro{citeindex}%
	\usebibmacro{journal}}
{\multicitedelim}
{\usebibmacro{postnote}}

\newcommand{\citeall}[1]{\citetitle{#1}, \cite{#1}, \citejournal{#1}}

\hsl{Prof. Dr. Wolfgang E. Nagel}
\betreuer{Dr. Elena-Ruxandra Cojocaru, Dr. Michael Bussmann, Matthias Werner}
\email{jonas.schenke@tu-dresden.de}
\newdate{date}{16}{04}{2017}
\date{\displaydate{date}}

\addbibresource{res/Referenzen.bib}

% Zu jedem Abschnitt Gliederung zeigen
\AtBeginSection[]
{
  \begin{frame}<beamer>{Gliederung}
    \tableofcontents[currentsection]
  \end{frame}
}

% auch bei subsections gliederung?
%\AtBeginSubsection[]
%{
%  \begin{frame}<beamer>{Gliederung}
%    \tableofcontents[currentsection,currentsubsection]
%  \end{frame}
%}

% bei sehr vielen sections, mehrspaltig
%\begin{frame}[shrink]
%\frametitle{Gliederung}
%\begin{multicols}{2}
%	\tableofcontents
%\end{multicols}
%\end{frame}

% alle Aufzaehlungen schrittweise zeigen
%\beamerdefaultoverlayspecification{<+->}

\begin{document}

\zihmaketitle

\begin{frame}{Aufgabenstellung}
\frametitle{Aufgabenstellung}
	\begin{itemize}
		\item Evaluierung und Performance-Analyse des derzeit fast durchgängig seriellen Wellenfrontrekonstruktionsalgorithmus
		\item Parallelisierung der kritischen Pfade für Mehrkernarchitekturen
		\item Performance-Messungen der parallelen Implementation
		\item Auswertung sowie Validierung der Ergebnisse
	\end{itemize}
\end{frame}

\begin{frame}{Gliederung}
  \setcounter{tocdepth}{1} 
  \tableofcontents
  % Die Option [pausesections] koennte nuetzlich sein.
\end{frame}



% im text korrekturen anzeigen
\newcommand{\correctme}[1]{\textcolor{red}{#1}}

% korrekturen ueber mehrere absaetze
\newenvironment{correctmore}{\color{red}}{\color{black}}

\lstset{
	backgroundcolor=\color{white},
	commentstyle=\color{green!40!black},
	frame=single,
	keywordstyle=\color{blue},
	breaklines=true,
	numbers=left,
	numbersep=5pt,
	tabsize=4,
	basicstyle=\footnotesize,
}


%TODO: 0. Einleitung --> Motivation % Hintegrund		1 -.    
%TODO: 1. Disclaimer									1  | 10 
%TODO: 2. Der Wellenfrontrekonstruktionsalgorithmus		4  |    
%TODO: 3. Performance									4 -'    
%TODO: 4. Optimierung									1 -.    
%TODO: 4.1 Parallelisierung								4  | 10 
%TODO: 4.2 Optimierung									5 -'    
%TODO: 5. wieder Performance							7 -.    
%TODO: 6. Ausblick										3 -' 10 

%TODO: 1 (0): 1
%TODO: 2 (1): 1
%TODO: 3 (2): 5.5
%TODO: 4 (3): 3.16
%TODO: 5 (4): 7.25
%TODO: 6 (5+6): 6

%TODO: mehr Motivation, Template Matching besser erklären? Pro Bildpaar mehr als 100 Sekunden, genauer auf Komplexität eingehen, Auslassen der Initialisierungsroutine erklären, iterativen Ansatz erklären

\section{Hinweise}

\begin{frame}{Hinweise}
	\begin{itemize}
		\item<2-> Teil des \textbf{Eu}ropean \textbf{C}luster of \textbf{A}dvanced \textbf{L}aser \textbf{L}ight Sources (EUCALL)-Projektes
		\rightarrow Überschneidung von \textbf{U}ltra\textbf{f}ast \textbf{D}ata \textbf{Ac}quisition (UFDAC; WP5) und \textbf{Pu}lse \textbf{C}haracterisation and \textbf{C}ontrol (PUCCA; WP7)
		\item<3-> Zusammenarbeit des \textbf{H}elmholtz-\textbf{Z}entrum \textbf{D}resden-\textbf{R}ossendorf e.V. (HZDR) und \textbf{E}uropean \textbf{S}ynchrotron \textbf{R}adiation \textbf{F}acillity (ESRF)
		\item<4-> \textbf{Algorithmus:} Dr. Sébastien Bérujon
		\item<5-> \textbf{Code:} Dr. Elena-Ruxandra Cojocaru
		\item<6-> \textbf{Daten:} Beamline BM05, ESRF, Grenoble, Frankreich
	\end{itemize}
\end{frame}
\section{Evaluierung des Wellenfrontrekonstruktionsalgorithmus}

\subsection{Überblick}
\begin{frame}{Versuchsaufbau}
	\begin{tikzpicture}
		\node at (0,0) {\includegraphics[width=0.98\linewidth]{./img/Versuchsaufbau.png}};
		\only<2>{\draw[red, very thick] (2.4, -2.2) rectangle (5.5, -0.2);}
		\only<3>{\draw[red, very thick] (1.2, -1.2) rectangle (1.6, -0.8);}
		\only<4>{\draw[red, very thick] (-1.1, -1.4) rectangle (0.3, 0);}
		
		\only<5>{\draw[red, very thick] (-5.5, 2.9) rectangle (-1.7, -1);}
		\only<6>{\draw[red, very thick] (-4.3, -2.7) rectangle (-2.1, -2);}
		\only<7>{\draw[red, very thick] (-3.3, -1) rectangle (-1.7, 0.1);}
		
		\only<8>{\draw[red, very thick] (-3.6, 1.2) rectangle (-1.9, 2.7);}
		\only<8>{\node at (3.4,0.8) {\includegraphics[width=0.5\linewidth]{img/ref_start0001_1-10}};}
		\only<8>{\draw[red, very thick] (0.7, 2.8) rectangle (6.0, -1.33);}
		\only<8>{\draw[red, very thick] (0.7, 2.8) -- (-1.9, 2.7);}
		\only<8>{\draw[red, very thick] (0.7, -1.33) -- (-1.9, 1.2);}
		
		\only<9>{\draw[red, very thick] (-5.3, -0.8) rectangle (-3.7, 0.3);}
		\only<10>{\draw[red, very thick] (-5.5, 1.5) rectangle (-3.8, 2.9);}
		
		\only<10>{\node at (3.4,0.8) {\includegraphics[width=0.5\linewidth]{img/E10001}};}
		\only<10>{\draw[red, very thick] (0.7, 2.8) rectangle (6.0, -1.33);}
		\only<10>{\draw[red, very thick] (0.7, 2.8) -- (-3.8, 2.9);}
		\only<10>{\draw[red, very thick] (0.7, -1.33) -- (-3.8, 1.5);}
	\end{tikzpicture}
	\nocite{Ber12}
	\nocite{Ber13}
	\nocite{Ber15}
	{\scriptsize \citeall{Ber15}}
\end{frame}

\begin{frame}{Überblick}
	\textbf{Hauptroutine:} 
	\begin{itemize}
		\item Korrigieren von Kamerafehlern
		\item Ablenkung nachverfolgen
		\item Wellenfront rekonstruieren
	\end{itemize}
\end{frame}

\begin{frame}{Hauptroutine}
	\only<1>{\vspace{-0.37cm}}
	\begin{tikzpicture}
		\node at (-5,0.6) {\includegraphics[width=0.4\linewidth]{pdf/graph_main}};
		\only<2-4>{\node at (-1,0) {\includegraphics[width=0.17\linewidth]{pdf/graph_speckle}}};
		
		\only<2>{\draw[red, very thick] (-6.0, -1.0) rectangle (-4.3, -0.3)};
		\only<2>{\draw[red, very thick] (-4.3, -0.3) -- (-2, 3.8)};
		\only<2>{\draw[red, very thick] (-4.3, -1.0) -- (-2, -3.8)};
		\only<2>{\draw[red, very thick] (-2, -3.8) rectangle (0.0, 3.8)};
		
		\only<3-4>{\draw[red!25, very thick] (-6.0, -1.0) rectangle (-4.3, -0.3)};
		\only<3-4>{\draw[red!25, very thick] (-4.3, -0.3) -- (-2, 3.8)};
		\only<3-4>{\draw[red!25, very thick] (-4.3, -1.0) -- (-2, -3.8)};
		\only<3-4>{\draw[red!25, very thick] (-2, -3.8) rectangle (0.0, 3.8)};
		
		\only<3>{\draw[red, very thick] (-1.95, 1.5) rectangle (-0.0, 2.05)};
		\only<3>{\node at (2,0) {\includegraphics[width=0.17\linewidth]{pdf/graph_first_pass}}};
		\only<3>{\draw[red, very thick] (0.0, 2.05) -- (1, 2.5)};
		\only<3>{\draw[red, very thick] (0.0, 1.5) -- (1, -2.5)};
		\only<3>{\draw[red, very thick] (1.0, -2.5) rectangle (3.1, 2.5)};
		
		\only<4>{\draw[red, very thick] (-1.95, -1.25) rectangle (-0.05, -0.55)};
		\only<4>{\node at (2,0) {\includegraphics[width=0.17\linewidth]{pdf/graph_second_pass}}};
		\only<4>{\draw[red, very thick] (-0.05, -0.55) -- (1, 2.5)};
		\only<4>{\draw[red, very thick] (-0.05, -1.25) -- (1, -2.5)};
		\only<4>{\draw[red, very thick] (1.0, -2.5) rectangle (3.1, 2.5)};
		
		\only<5>{\draw[red, very thick] (-6.0, -1.0) rectangle (-4.3, -0.3)};
		\only<5>{\node at (1,2.2) {\includegraphics[width=0.38\linewidth]{img/SpeckDisH_E10001_edf_ref_start0001_1-10_edf}}};
		\only<5>{\node at (1,-1.5) {\includegraphics[width=0.38\linewidth]{img/SpeckDisV_E10001_edf_ref_start0001_1-10_edf}}};
		\only<5>{\draw[red, very thick] (-1.2, -3.5) rectangle (2.9, 3.9)};
		\only<5>{\node at (0.8, 0.5) {Horizontaler Gradient}};
		\only<5>{\node at (0.8, -3.3) {Vertikaler Gradient}};
		\only<5>{\draw[red, very thick] (-4.3, -0.3) -- (-1.2, 3.9)};
		\only<5>{\draw[red, very thick] (-4.3, -1.0) -- (-1.2, -3.5)};
		
		\only<6-7>{\node at (0.4, -2.5) {\footnote{\citeall{FC88}}};}
		\only<6-7>{\draw[red, very thick] (-6.0, -1.1) rectangle (-4.3, -1.8)};
		\only<6>{\node at (-1,0) {\includegraphics[width=0.17\linewidth]{pdf/graph_fc}}};
		\only<6>{\draw[red, very thick] (-4.3, -1.1) -- (-2, 2.5)};
		\only<6>{\draw[red, very thick] (-4.3, -1.8) -- (-2, -2.5)};
		\only<6>{\draw[red, very thick] (-2, -2.5) rectangle (0.0, 2.5)};
		
		\only<7>{\node at (0.5,1) {\includegraphics[width=0.6\linewidth]{img/2D_E10001_edf_ref_start0001_1-10_edf}}};
		\only<7>{\draw[red, very thick] (-2.7, -2) rectangle (3.4, 3.2)};
		\only<7>{\node at (0.7, -1.7) {Integrierte Gradientenmatrix}};
		\only<7>{\draw[red, very thick] (-4.3, -1.1) -- (-2.7, 3.2)};
		\only<7>{\draw[red, very thick] (-4.3, -1.8) -- (-2.7, -2)};
	\end{tikzpicture}
\end{frame}
\chapter{Performance-Analyse der derzeitigen Implementierung}

\section{Komplexität}

\subsection{Speckle-Tracking}

Der erste Schritt des Speckle-Trackings ist die Feststellung der starren Verschiebung. Werden hierfür feste Werte angenommen, ist diese Komplexität konstant. Wird allerdings ein Korrelationsverfahren verwendet, so ist die Komplexität $\mathcal{O}(\gls{resolution} \cdot log(\gls{resolution}))$ für die \glsfirst{resolution} \glssymbol{resolution}, da hierfür \glspl{FFT} eingesetzt werden. 

Der nächste Verarbeitungsschritt ist der erste Durchlauf. Hier werden die Eingabebilder zunächst durch die Selektion der \gls{ROI}, also auf die \glsfirst{rroi} \glssymbol{rroi} verkleinert und anschließend in Blöcke mit konstanter Größe aufgeteilt, welche dann ineinander mittels des Template-Matchings gesucht werden. Die Komplexität der einzelnen Suchvorgänge kann somit auch als konstant angesehen werden. Der Durchlauf braucht demzufolge $\mathcal{O}(\gls{rroi})$ Schritte. Die anschließende Interpolation wird auf alle Pixel des Bildes angewandt und hat demzufolge eine Komplexität von $\mathcal{O}(\gls{rroi})$.

Der zweite Durchlauf, der als nächstes folgt, werden Subbilder, wie \ref{sec:speckle-tracking} beschrieben, generiert. durch die \glsfirst{uap} \glssymbol{uap} wird \gls{rroi}, und damit auch die Komplexität dieses Teils, mit dem Faktor $\gls{uap}^2$ verringert. Die \glsfirst{corrsize} \glssymbol{corrsize} legt nimmt auf das Template-Matching Einfluss, dessen Komplexität nun bei $\mathcal{O}(\gls{corrsize} \cdot log(\gls{corrsize}))$ liegt. Der Einfluss der \glsfirst{gridResol} \glssymbol{gridResol} ist, ähnlich zu \gls{uap}, eine Verringerung mit dem Faktor $\gls{gridResol}^2$. Die Gesamtkomplexität der Template-Matchings im zweiten Durchlauf liegt somit bei:

\begin{center}
	$\mathcal{O}(\frac{\gls{rroi} \cdot \gls{corrsize} \cdot log(\gls{corrsize})}{(\gls{uap}^2 \cdot \gls{gridResol})^2})$
\end{center}

Für die auf den Template-Matching-Prozess folgende Subpixel-Interpolierung werden neun Pixel in der Umgebung des Maximums jedes Übereinstimmungsmatrix interpoliert, womit dieser Schritt folgende Komplexität aufweist:

\begin{center}
	$\mathcal{O}(\frac{\gls{rroi}}{(\gls{uap}^2 \cdot \gls{gridResol}^2)})$
\end{center}

Am Ende des Speckle-Tracking-Algorithmus wird versucht, nicht zuordenbare Ergebnisse mit einer anderen \glsfirst{corrsize} \glssymbol{corrsize} erneut zuzuordnen. Im schlimmsten Fall wird der zweite Durchlauf für die Hälfte der Subbilder \gls{ncorr}-fach wiederholt, wobei \gls{ncorr} die \glsdesc{ncorr} repräsentiert. Bei höheren Fehlerraten über 50\% bricht das Programm. In der hier als Grundlage vorliegenden Implementierung ist $\gls{ncorr} = 6$.

Die Gesamtkomplexität des Speckle-Tracking-Algorithmus liegt damit in der Komplexitätsklasse:

\begin{center}
	$\mathcal{O}(\frac{\gls{rroi} \cdot \gls{corrsize} \cdot log(\gls{corrsize})}{(\gls{uap}^2 \cdot \gls{gridResol})^2})$
\end{center}

\subsection{Gradientenintegration}

Um eine effiziente Integration der Gradienten zu ermöglichen, beruht der von \citeauthor{FC88} vorgeschlagene Algorithmus auf der Integration im Frequenzraum. Hierzu werden zuerst die Gradientenbilder mittels \glspl{FFT} in diesen Raum transformiert, dort in linearer Komplexität integriert und zum Schluss wieder zurück transformiert. 
Aufgrund der Verwendung von \glspl{FFT} befindet sich dieser Algorithmus in der Komplexitätsklasse:

\begin{center}
	$\mathcal{O}(\gls{resolution} \cdot log(\gls{resolution}))$
\end{center}

\subsection{Kalibrierung}

Die Komplexität der Parameterinitialisierung innerhalb der Kalibrierungsroutine ist trivial und kann daher als linear angenommen werden. 

Während der Dunkelfeldkalibrierung wird ein Medianbild aus \gls{N} Bildern erstellt. Für diese \gls{N} Bilder wird an jeder Position des Bildes der Median gebildet. Somit ist dieser Teil des Algorithmus in der Komplexitätsklasse:

\begin{center}
	$\mathcal{O}(\gls{N} \cdot \gls{resolution})$
\end{center}

Die Nullfeldkalibrierung verläuft sehr ähnlich. Hier wird ein Durchschnittsbild aus \gls{N} Bildern erstellt, für welches alle Pixel  der \gls{N} Bilder an der selben Position aufaddiert und anschließend durch \gls{N} geteilt werden. Somit ist auch dieser Teil des Algorithmus in der Komplexitätsklasse:

\begin{center}
	$\mathcal{O}(\gls{N} \cdot \gls{resolution})$
\end{center}

Nachdem diese Kalibrierungen abgeschlossen sind, folgt die Hauptschleife in der die Sensorstreuung mittels des Speckle-Tracking-Algorithmus aus allen möglichen Bildpaarkombinationen $\gls{N}^2$ berechnet wird. Hierzu werden zuerst die Sensorbilder mittels der bei der Kalibrierung ermittelten Werte mit einer Komplexität von $\mathcal{O}(\gls{resolution})$ korrigiert und anschließend bestimmt der Speckle-Tracking-Algorithmus die Streuung des Sensors. Die gesamte Streueffeckterkennung hat somit eine Komplexität von: 

\begin{center}
	$\mathcal{O}(\frac{\gls{N}^2 \cdot\gls{rroi} \cdot \gls{corrsize} \cdot log(\gls{corrsize})}{(\gls{uap}^2 \cdot \gls{gridResol})^2})$
\end{center}

Dies liegt insbesondere für kleine \gls{uap} und \gls{gridResol} in der Komplexitätsklasse:

\begin{center}
	$\mathcal{O}(\gls{N}^2 \cdot\gls{rroi} \cdot \gls{corrsize} \cdot log(\gls{corrsize}))$
\end{center}

Da die Streueffekterkennung deutlich komplexer als der Rest der Kalibrierungsphase ist, liegt die Komplexität dieser ebenfalls in der oben beschriebenen Komplexitätsklasse. 

\subsection{Hauptroutine}

Ähnlich zur Kalibrierung beginnt die Hauptroutine mit einer linearen Parameterinitialisierung. Auf diese folgt die Hauptschleife, welche für die \gls{N_Paare} jeweils ein mal ausgeführt wird. Genau, wie in der Kalibrierung wird auch hier wieder eine Korrektor der Eingabebilder in linearer Zeit vorgenommen. Dies wird gefolgt vom Speckle-Tracking-Algorithmus und der Gradientenintegration. Die höchste Komplexität hat hier auch wieder der Template-Matching-Algorithmus, welcher damit die Komplexitätsklasse und somit auch der gesamten Hauptroutine festlegt auf:

\begin{center}
	$\mathcal{O}(\frac{\gls{N_Paare} \cdot\gls{rroi} \cdot \gls{corrsize} \cdot log(\gls{corrsize})}{(\gls{uap}^2 \cdot \gls{gridResol})^2})$
\end{center}

Dies liegt, ähnlich zur Kalibrierung, für kleine \gls{uap} und \gls{gridResol} in der Komplexitätsklasse:

\begin{center}
	$\mathcal{O}(\gls{N_Paare} \cdot\gls{rroi} \cdot \gls{corrsize} \cdot log(\gls{corrsize}))$
\end{center}

\section{Benchmark}

\subsection{Testsystem und Laufzeitumstände}

\paragraph{Testsystem}

Alle Benchmarks liefen auf den \textit{haswell}-Partition des Taurus-Supercomputers an der Technischen Universität Dresden. Jeder Knote dieser Partition ist ausgestattet mit zwei Intel\textregistered Xeon\textregistered E5-2680 v3 \glspl{CPU}. Diese haben zwölf Rechenkerne, die mit bis zu 2.50 GHz getaktet sind. MultiThreading war hierbei nicht aktiviert. Die Knoten haben 64 GiB (\textit{haswell64}), 128 GiB (\textit{haswell128}) oder 256 GiB (\textit{haswell256}) Arbeitsspeicher zur Verfügung. Zusätzlich ist pro Rechenknoten eine 128 GB \gls{SSD} installiert. Es wurde unter anderem Python 2.7.11 mit numpy 1.10.1 und OpenCV 3.1.0 verwendet. Eine komplette Liste aller geladenen Module lässt sich auf dem GitHub-Repository dieses Projektes\footnote{\url{https://github.com/ComputationalRadiationPhysics/Wavefront-Sensor/blob/9e1c91790a5461f3ff1b6e4b6f6022529b88cc34/loaded_libs.txt}} finden.

\paragraph{Laufzeitumstände}

Jede Konfiguration, bestehend aus Datensatz und Kernanzahl, wurde nach vier Aufwärmiterationen fünf mal ausgeführt. Hierbei wurden jeweils die reinen Ausführungszeiten des gesamten Skripts und einzelner Funktionen erfasst. Aus allen vorliegenden Zeiten wurde \gls{IO}-Zeiten herausgerechnet. Die Laufzeit mit den entsprechenden Datensätzen wurde auf unterschiedlich vielen Kernen von eins bis 24 gemessen. Jeder Benchmark lief exklusiv auf einem Knoten. 

\paragraph{Datensätze}

Zur Leistungsfeststellung der vorliegenden Implementierung werden drei verschiedene Arten von Datensätzen verwendet: \textit{detectorDistortion}, \textit{Experiment 6}, \textit{Lenses}. \textit{detectorDistortion} wird zur Kalibrierung verwendet. Die Eigenschaften dieser Typen werden in Tabelle \ref{tab:datasets} gegenüber gestellt.  Von diesem Datensatztyp gibt es einen Datensatz mit 25 Bildern. Es existieren drei \textit{Experiment 6} Datensätze mit 21 (\textit{Experiment 6 Lenses 200}), 11 (\textit{Experiment 6 Lenses 500}) und 14 Bildpaaren (\textit{Experiment 6 Lenses 1500}). Vom letzten Typ existieren vier Datensätze mit jeweils zehn (\textit{Lenses Set 1}), fünf (\textit{Lenses Set 2}), zwei (\textit{Lenses Set 3}) und einem Bildpaar.

\begin{table}
	\begin{tabularx}{\textwidth}{| X || X || X | X |}
		\hline
		& \textbf{Kalibrierung} & \multicolumn{2}{c|}{\textbf{Hauptroutine}} \\
		\cline{2-4}
		& detectorDistortion & Experiment 6 & Lenses \\
		\hline
		\hline
		\gls{rroi} (in Pixel) & 1848 x 1848 & Sensor 1: 550 x 550 \newline
		Sensor 2:1450 x 1450  & 1450 x 1550 \\
		\hline
		\gls{gridResol} & 4 & 1 & 1 \\
		\hline
		\gls{corrsize} & 37 & 91 & 41 \\
		\hline
		\gls{uap} & 1 & 1 & 1 \\
		\hline
		Pixelgröße & gleich & unterschiedlich & gleich \\
		\hline
	\end{tabularx}
	\caption{Parameter der Datensätze}
	\label{tab:datasets}
\end{table}

\subsection{Laufzeiten}

Die Laufzeiten der Konfigurationen, dargestellt in Abbildung \ref{fig:gesamtlaufzeiten},  variieren untereinander stark und reichen von ca. dreieinhalb Stunden für den \textit{Lenses Set 1}-Datensatz auf einem Kern bis hin zu ca. vier Minuten für den \textit{Lenses Set 3} Datensatz mit einem Bild auf 24 Kernen.

\begin{center}
	\begin{figure}[htbp]
		\begin{subfigure}[b]{0.325\textwidth}
			\centering
			\includegraphics[width=\textwidth]{pdf/times_detector_distortion}
			\caption[Detector Distortion]{Detector Distortion}
			\label{fig:times_det_dist}
		\end{subfigure}
		\hfill
		\begin{subfigure}[b]{0.325\textwidth}
			\centering
			\includegraphics[width=\textwidth]{pdf/times_exp6}
			\caption[Experiment 6]{Experiment 6}
			\label{fig:times_exp6}
		\end{subfigure}
		\hfill
		\begin{subfigure}[b]{0.325\textwidth}
			\centering
			\includegraphics[width=\textwidth]{pdf/times_lenses}
			\caption[Lenses]{Lenses}
			\label{fig:times_lenses}
		\end{subfigure}
		\caption{Gesamtlaufzeiten}
		\label{fig:gesamtlaufzeiten}
	\end{figure}
\end{center}

Der Speedup des Programmes skaliert mit der Anzahl der Prozessorkerne nicht linear und flacht schnell ab. Die Speedup-Faktoren liegen für die \textit{Detector Distortion} und \textit{Experiment 6} Datensätze übersteigt fünf nicht. Bei den \textit{Lenses}-Datensätzen hingegen wird bei 24 Kernen ein Speedup von mehr als zehn erreicht. In den auf Abbildung \ref{fig:speedup} visualisierten Graphen ist eine starke Skalierung deutlich erkennbar. 

\begin{center}
	\begin{figure}[htbp]
		\begin{subfigure}[b]{0.325\textwidth}
			\centering
			\includegraphics[width=\textwidth]{pdf/speedup_detector_distortion}
			\caption[Detector Distortion]{Detector Distortion}
			\label{fig:speedup_det_dist}
		\end{subfigure}
		\hfill
		\begin{subfigure}[b]{0.325\textwidth}
			\centering
			\includegraphics[width=\textwidth]{pdf/speedup_exp6}
			\caption[Experiment 6]{Experiment 6}
			\label{fig:speedup_exp6}
		\end{subfigure}
		\hfill
		\begin{subfigure}[b]{0.325\textwidth}
			\centering
			\includegraphics[width=\textwidth]{pdf/speedup_lenses}
			\caption[Lenses]{Lenses}
			\label{fig:speedup_lenses}
		\end{subfigure}
		\caption{Speedup}
		\label{fig:speedup}
	\end{figure}
\end{center}

Um Engpässe und besonders rechenaufwendige Funktionen zu identifizieren, wurde das Programm mit Zeitmessern versehen, die Ausführungszeiten und Aufrufanzahl protokolliert haben. Ein Überblick über das Gesamtprogramm ist in Abbildung \ref{fig:perc_main} zu sehen.

\begin{center}
	\begin{figure}[htbp]
		\centering
		\includegraphics[width=0.7\textwidth]{pdf/main}
		\caption{Anteile der Laufzeiten}
		\label{fig:perc_main}
	\end{figure}
\end{center}

Hierbei ist eindeutig zu sehen, dass die meiste Zeit für das Speckle-Tracking benötigt wird. um weitere Informationen über die Laufzeiten der einzelnen Speckle-Tracking-Schritte zu gewinnen, wurde dieses ebenfalls mit Zeitmessern versehen. Die zeitliche Aufteilung dieser zeigt in Abbildung \ref{fig:perc_speckle}, dass hierbei der zweite Durchlauf am meisten Zeit benötigt. 

\begin{center}
	\begin{figure}[htbp]
		\centering
		\includegraphics[width=0.7\textwidth]{pdf/speckle}
		\caption{Anteile der Laufzeiten des Speckle-Tracking-Algorithmus}
		\label{fig:perc_speckle}
	\end{figure}
\end{center}

Die kumulative Zeit der fünf rechenaufwendigsten Funktionen aller Konfigurationen, dargestellt in Abbildung \ref{fig:perc_slow}, liegt jeweils bei über 95\% der Gesamtzeit. 

\begin{center}
	\begin{figure}[htbp]
		\centering
		\includegraphics[width=0.7\textwidth]{pdf/slow}
		\caption{Anteile der Laufzeiten der langsamsten Funktionen}
		\label{fig:perc_slow}
	\end{figure}
\end{center}

\section{Grund der Performance-Engpässe}

Der Grund der langen Rechenzeiten des Template-Matchings und der Subpixel-Interpolation liegt in der hohen Anzahl der Aufrufe dieser begründet. Der zweite Durchlauf allein wird im \textit{Experiment 6 Lenses 200}-Datensatz über 5.3 Millionen mal aufgerufen. In jedem dieser Aufrufe wird das Template-Matching und die Subpixel-Interpolation jeweils ein mal genutzt. Hinzu kommt, dass, bis auf das Template-Matching, der zweite Durchlauf nur geringen Gebrauch von bereits optimierten Bibliotheken wie numpy macht und somit der Pyhton-Overhead hinzukommt. 

Innerhalb des Speckle-Trackings ist der Aufruf des zweiten Durchlaufes mittels der joblib parallelisiert. Diese nutzt standardmäßig die multiprocessing-Bibliothek, welche für jeden Thread einen Fork der gesamten Python-Umgebung erstellen muss, was zu einem erheblichen Overhead führt\footnote{\url{https://pythonhosted.org/joblib/parallel.html}}.

Die hohe Rechenzeit der Gradienten-Integration ist im Aufruf dieser auf die Größe des Gesamtbildes begründet.

Insgesamt hat das Programm eine schlechte CPU-Auslastung \correctme{nachweisen mittels time Befehl!}, wodurch häufig einige Kerne nicht oder nur wenig genutzt werden. 
\chapter{Parallelisierung der kritischen Pfade}

\correctme{ - iterativer Ansatz}

\section{Kompilieren}

\begin{correctmore}
	- kompilieren des kompletten Projektes mit Cython
	- funktioniert, aber Ergebnisse bringen nicht gewünschten Speedup bzw. nur manchmal
	--> Ansatz verworfen
\end{correctmore}

\section{Parallelisierung}

\subsection{Parallelisierung der Verarbeitung einzelner Bildpaare mittels MPI}

\begin{correctmore}
	- paralleles Berechnen mehrerer Bildpaare auf verschiedenen Nodes
	- simpelster Ansatz
	- linearer Speedup erwartet
	- Nachteil: schwache Skalierung --> es müssen genug Bildpaare vorhanden sein
\end{correctmore}

\subsection{Parallelisierung innerhalb der Verarbeitung einzelner Bildpaare mittels MPI}

\begin{correctmore}
	- Ersetzen der joblib mit MPI
	- Parallelisieren der Fehlerkorrektur
	- Nachteil:
		-> keine allzu hohe performancesteigerung erwartet
	- Vorteil:
		-> Profiling mittels spezieller Tools möglich
\end{correctmore}

\section{Optimierung der Python-Engpässe}

\begin{correctmore}
	- Optimieren einzelner in Python implementierter Programmteile, die sich als besonders langsam herausgestellt haben
\end{correctmore}

\subsection{Nutzen von Python Intrinsics}

\begin{correctmore}
	- Ersetzen des Codes mit bereits in Python/Cython/numpy/OpenCV implementierter Funktionen
	- entfernt Interpreter-Aufwand ein wenig
	- grundlegenste Optimierung
\end{correctmore}

\subsection{Decorators}

\correctme{ - kompilieren einzelne Funktion}

\subsubsection{numba}

\correctme{ - just in time compiler}

\subsubsection{Cython}

\begin{correctmore}
	- regulärer Compiler
	- weitere Optimierungsmöglichkeiten (z.B. mit C Typen)
\end{correctmore}

\subsection{numpy}

\correctme{ - SIMD Operationen auf Datenarray}

\subsection{numexpr}

\correctme{ - SIMD Operationen auf Datenblöcke}








\iffalse

\section{Lösungsstrategien}

\subsection{Primitive Aufteilung auf mehrere Nodes}

- Kurzbeschreibung: kompletten Algorithmus auf mehreren Nodes gleichzeitig ausführen
- Pro: hoher Speedup trotz simpler Implementierung
- Contra: immer noch größtenteils seriell auf einzelnen Nodes

\subsection{Jobserver für rechenaufwendigen Funktionen}

- Kurzbeschreibung: Jobserver für Rechenaufwendige Funktionen erstellen und diese in Auftrag geben, sobald ein Ergebnis benötigt wird
- Pro: effiziente Verteilung
- Contra: Main-Threads warten trotzdem noch

\subsection{Primitive Aufteilung auf mehrere Nodes mit Jobservern}

- Kurzbeschreibung: Primitive Aufteilung (wie bei 3.1.1) mit Jobservern auf jedem Node (wie bei 3.2.2)
- Pro: höhere parallele Auslastung der einzelen Nodes
- Contra: immer noch nicht optimal durch Kopieroverhead bei jedem einzelnen Frame

\subsection{Paketweises Abarbeiten der Daten}

- Kurzbeschreibung: Framepakete werden auf mehrere Nodes aufgeteilt, wo diese in SIMD Art und Weise abgearbeitet werden. Rechenaufwendige Funktionen werden parallelisiert und ausgelagert (durch SIMD kein Jobserver mehr nötig)
- Pro: kein Jobserver nötig, wenig Overhead durch kopieren
- Contra: erhöhter Code-Portierungsaufwand

\section{Implementierung}

\subsection{Implementierte Lösungsstrategie}

- 3.1.4 wird implementiert
-> geringster Overhead
-> höchste Performance Erwartung
-> hohe Auslastung der Nodes
-> einfache Aufteilung auf mehrere Nodes mittels OpenMPI in Python
-> Auslagerung rechenaufwendiger Funktionen relativ simpel möglich (C/C++ Interface)
ABER: Umschreiben des jetzigen Codes in großen Teilen nötig
- Möglichkeiten und Limitierungen:
- Möglichkeiten:
-> Portierung auf weitere Architekturen (CUDA/OpenCL/...) ohne große Änderung des bestehenden Codes möglich
-> lineare Skalierung mit steigender Anzahl der Nodes
-> gut skalierbar mit Anzahl der Nodes (ohne Änderung des Codes)
- Limitierungen:
-> nur ganze Frame-Pakete können verarbeitet werden
-> Synchronisation nach Verarbeitung kann Gesamtprozess ein wenig ausbremsen
-> mögliches Bottleneck beim Datenaustausch zwischen den Nodes (z.B. über Ethernet)
-> Ringpuffer zum Sammeln und Zwischenspeichern der Daten vor der Verarbeitung notwendig --> hohe RAM Anforderungen
- Mindestanforderungen:
-> korrekte Verarbeitung der Daten
-> nahezu lineare Skalierung
-> Erweiterbarkeit der Hardware ohne Änderung des Quellcodes
-> 2048*2048*10fps (?)

\subsection{Softwareimplementierung}

- Implementierung vorstellen

\chapter{Parallelisierung der kritischen Pfade}

\section{Lösungsstrategien}

Die Ergebnisse der Performance-Analyse legen nahe, dass der größtmögliche Performance-Anstieg durch Parallelisierung erreicht werden kann. Hierzu bieten sich verschiedene Ansätze, die im folgenden genauer beleuchtet werden sollen. 

\subsection{Primitive Aufteilung der Frames auf mehrere Nodes}
\label{subsec:primitive_splitting}

Die einfachste Art der Parallelisierung besteht darin die Hauptroutine auf mehreren Nodes gleichzeitig zu starten und mit diesen einzelne Frames parallel auszuführen, wobei der Algorithmus an sich dabei unverändert bleibt. 
Am Ende werden dann die Ergebnisse der einzelnen Nodes wieder zusammengeführt.

Vorteil hierbei ist eine relativ einfache Implementierung. 

Nachteilig hingegen ist die geringe Flexibilität und das Vernachlässigen des verbleibenden Potentials der Nodes. Des weiteren entsteht ein Overhead durch den Funktionsaufruf bei jedem einzelnen Frame. 

%TODO: pro and contra

\subsection{Jobserver für rechenaufwendigen Funktionen}
\label{subsec:jobserver}

Eine weitere Möglichkeit besteht in der Einführung eines Jobservers, der eine vorgefertigte Menge an rechenaufwendigen Funktionen an mehrere Nodes verteilt und das Ergebnis anschließend zurückgibt. 

%TODO: pro and contra

\subsection{Primitive Aufteilung auf mehrere Nodes mit Jobservern}
 - Frame Pakete erstellen und an einzelne Nodes verschicken, wo diese dann seriell ausgeführt werden und rechenaufwendige Funktionen an einen Jobserver übergeben

%TODO: pro and contra

\subsection{Paketweises Abarbeiten der Daten}
 - Frame Pakete erstellen und auf Nodes verteilen, wo diese dann schrittweise parallel verarbeitet werden und das Paket als ein ganzes Rechenaufwendige Funktionen erreicht. Diese werden dann gleichzeitig parallel verarbeitet. 

%TODO: pro and contra

\section{Implementierung}

\subsection{Implementierte Lösungsstrategie}

\subsection{Softwareimplementierung}
\fi
\chapter{Performance-Messungen der parallelen Implementation}

\section{Evaluierung der Optimierungen}

\subsection{Parallelisierung}

\begin{correctmore}
	- Speedup: 2-5x (geschätzt; ohne mehr nodes)
	- linear mit Anzahl der Nodes bis Anzahl der Bildpaare
\end{correctmore}

\subsection{Optimierung von Python Engpässen}

\subsubsection{Intrinsics}

\begin{correctmore}
	TODO
\end{correctmore}

\subsubsection{Kompilieren}

\begin{correctmore}
TODO
\end{correctmore}

\section{Einfluss der Parameter}

\begin{correctmore}
	- Algorithmus in verschiedenen Konfigurationen benchmarken (Framepaket Größe variieren, ...)
	- Performance Sweet-Spot finden
\end{correctmore}

\section{Skalierung}

\subsection{Skalierungsfaktor}

\correctme{ - auf Skalierungsfaktor eingehen und diesen in Bezug zu Parametern und verwendeter Hardware setzen}

\subsection{Sättigung}

\begin{correctmore}
	- Sättigungspunkt -grund beschreiben
	- Sättigungsgrund:
	-> Framepaketgröße: Rechenaufwand >> Kopieraufwand -> keine weitere Optimierung des Kopieraufwandes möglich
	-> Anzahl der Nodes: Kopieraufwand übersteigt Rechenaufwand bzw. nähert sich an
	-> nur begrenzte Menge an Datenverfügbar
\end{correctmore}
\section{Stand der Arbeit}

\subsection{Jetziger Stand}
\begin{frame}{Jetziger Stand}
\correctme{is doch eh alles falsch}
	\textbf{Bereits erledigt:}
	\begin{itemize}
		\item Evaluierung
		\item Performance-Analyse
		\item Kompilieren
	\end{itemize}
	
	\begin{uncoverenv}<2->
		\textbf{In Bearbeitung:}
		\begin{itemize}
			\item Parallelisieren
			\item Optimieren von reinem Python-Code
		\end{itemize}
	\end{uncoverenv}
\end{frame}

\subsection{Weiterer Plan}
\begin{frame}{Weiterer Plan}
\textbf{Noch geplant:}
\begin{itemize}
	\item Optimierung kritischer Pfade
	\begin{itemize}
		\item vermehrte Verwendung optimierter Bibliotheken
		\item C/C++ Portierung einzelner Funktionen
	\end{itemize}
	\item Validierung der parallelen Implementierung
	\item Performance-Messung der parallelen Implementierung
\end{itemize}
\end{frame}

\subsection{Probleme}
\begin{frame}{Probleme}
	\begin{itemize}
		\item Bugs in Referenzimplementierung
		\item<2-> Python Multithreading
		\begin{itemize}
			\item<3-> viele Profiler für Python, wenige mit Multithreading-Support
			\item<4-> Forken der kompletten Python-Umgebung nötig
		\end{itemize}
		\item<5-> schlechte Kompilierergebnisse
	\end{itemize}
\end{frame}
\appendix
\section<presentation>*{\appendixname}
\subsection<presentation>*{Weiterführende Literatur}

\begin{frame}[allowframebreaks]
  \frametitle<presentation>{Weiterführende Literatur}
    
 	{\footnotesize
 		
  	\printbibliography}
  
\end{frame}

\end{document}
