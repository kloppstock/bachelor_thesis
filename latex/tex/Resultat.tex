\chapter{Analyse sowie Validierung der Ergebnisse}

\section{Speedup}

\begin{correctmore}
	- Ergebnisse werten (im Vergleich zu seriell)
	-> parallele Implementierung deutlich schneller dank hoher Auslastung der Nodes und Aufteilen der Frames in Framepakete, welche parallel verarbeitet werden
\end{correctmore}
\section{Wertung des Ergebnisses}

\begin{correctmore}
	- Analyse des erreichten Ergebnisses mit Rückblick auf die Möglichkeiten (aus 3.2)
	- Analyse des noch nicht genutzten Potentials mit Rückblick auf die Möglichkeiten (aus 3.2)
\end{correctmore}

\section{Verbesserungsmöglichkeiten}

\begin{correctmore}
	-> Anpassung des template-matching Prozesses (nur Punkt mit größter Übereinstimmung suchen mittels Runterskalierung)
	-> Parallelisierung von detectorDistorion.py
	-> Nutzen von FFTW für Frankot\_chellappa
	-> Implementierung der rechenaufwendigen Funktionen für GPGPUs
\end{correctmore}