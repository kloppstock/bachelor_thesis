\chapter{Parallelisierung der kritischen Pfade}

\correctme{ - iterativer Ansatz}

\section{Kompilieren}

\begin{correctmore}
	- kompilieren des kompletten Projektes mit Cython
	- funktioniert, aber Ergebnisse bringen nicht gewünschten Speedup bzw. nur manchmal
	--> Ansatz verworfen
\end{correctmore}

\section{Parallelisierung}

\subsection{Parallelisierung der Verarbeitung einzelner Bildpaare mittels MPI}

\begin{correctmore}
	- paralleles Berechnen mehrerer Bildpaare auf verschiedenen Nodes
	- simpelster Ansatz
	- linearer Speedup erwartet
	- Nachteil: schwache Skalierung --> es müssen genug Bildpaare vorhanden sein
\end{correctmore}

\subsection{Parallelisierung innerhalb der Verarbeitung einzelner Bildpaare mittels MPI}

\begin{correctmore}
	- Ersetzen der joblib mit MPI
	- Parallelisieren der Fehlerkorrektur
	- Nachteil:
		-> keine allzu hohe performancesteigerung erwartet
	- Vorteil:
		-> Profiling mittels spezieller Tools möglich
\end{correctmore}

\section{Optimierung der Python-Engpässe}

\begin{correctmore}
	- Optimieren einzelner in Python implementierter Programmteile, die sich als besonders langsam herausgestellt haben
\end{correctmore}

\subsection{Nutzen von Python Intrinsics}

\begin{correctmore}
	- Ersetzen des Codes mit bereits in Python/Cython/numpy/OpenCV implementierter Funktionen
	- entfernt Interpreter-Aufwand ein wenig
	- grundlegenste Optimierung
\end{correctmore}

\subsection{Decorators}

\correctme{ - kompilieren einzelne Funktion}

\subsubsection{numba}

\correctme{ - just in time compiler}

\subsubsection{Cython}

\begin{correctmore}
	- regulärer Compiler
	- weitere Optimierungsmöglichkeiten (z.B. mit C Typen)
\end{correctmore}

\subsection{numpy}

\correctme{ - SIMD Operationen auf Datenarray}

\subsection{numexpr}

\correctme{ - SIMD Operationen auf Datenblöcke}








\iffalse

\section{Lösungsstrategien}

\subsection{Primitive Aufteilung auf mehrere Nodes}

- Kurzbeschreibung: kompletten Algorithmus auf mehreren Nodes gleichzeitig ausführen
- Pro: hoher Speedup trotz simpler Implementierung
- Contra: immer noch größtenteils seriell auf einzelnen Nodes

\subsection{Jobserver für rechenaufwendigen Funktionen}

- Kurzbeschreibung: Jobserver für Rechenaufwendige Funktionen erstellen und diese in Auftrag geben, sobald ein Ergebnis benötigt wird
- Pro: effiziente Verteilung
- Contra: Main-Threads warten trotzdem noch

\subsection{Primitive Aufteilung auf mehrere Nodes mit Jobservern}

- Kurzbeschreibung: Primitive Aufteilung (wie bei 3.1.1) mit Jobservern auf jedem Node (wie bei 3.2.2)
- Pro: höhere parallele Auslastung der einzelen Nodes
- Contra: immer noch nicht optimal durch Kopieroverhead bei jedem einzelnen Frame

\subsection{Paketweises Abarbeiten der Daten}

- Kurzbeschreibung: Framepakete werden auf mehrere Nodes aufgeteilt, wo diese in SIMD Art und Weise abgearbeitet werden. Rechenaufwendige Funktionen werden parallelisiert und ausgelagert (durch SIMD kein Jobserver mehr nötig)
- Pro: kein Jobserver nötig, wenig Overhead durch kopieren
- Contra: erhöhter Code-Portierungsaufwand

\section{Implementierung}

\subsection{Implementierte Lösungsstrategie}

- 3.1.4 wird implementiert
-> geringster Overhead
-> höchste Performance Erwartung
-> hohe Auslastung der Nodes
-> einfache Aufteilung auf mehrere Nodes mittels OpenMPI in Python
-> Auslagerung rechenaufwendiger Funktionen relativ simpel möglich (C/C++ Interface)
ABER: Umschreiben des jetzigen Codes in großen Teilen nötig
- Möglichkeiten und Limitierungen:
- Möglichkeiten:
-> Portierung auf weitere Architekturen (CUDA/OpenCL/...) ohne große Änderung des bestehenden Codes möglich
-> lineare Skalierung mit steigender Anzahl der Nodes
-> gut skalierbar mit Anzahl der Nodes (ohne Änderung des Codes)
- Limitierungen:
-> nur ganze Frame-Pakete können verarbeitet werden
-> Synchronisation nach Verarbeitung kann Gesamtprozess ein wenig ausbremsen
-> mögliches Bottleneck beim Datenaustausch zwischen den Nodes (z.B. über Ethernet)
-> Ringpuffer zum Sammeln und Zwischenspeichern der Daten vor der Verarbeitung notwendig --> hohe RAM Anforderungen
- Mindestanforderungen:
-> korrekte Verarbeitung der Daten
-> nahezu lineare Skalierung
-> Erweiterbarkeit der Hardware ohne Änderung des Quellcodes
-> 2048*2048*10fps (?)

\subsection{Softwareimplementierung}

- Implementierung vorstellen

\chapter{Parallelisierung der kritischen Pfade}

\section{Lösungsstrategien}

Die Ergebnisse der Performance-Analyse legen nahe, dass der größtmögliche Performance-Anstieg durch Parallelisierung erreicht werden kann. Hierzu bieten sich verschiedene Ansätze, die im folgenden genauer beleuchtet werden sollen. 

\subsection{Primitive Aufteilung der Frames auf mehrere Nodes}
\label{subsec:primitive_splitting}

Die einfachste Art der Parallelisierung besteht darin die Hauptroutine auf mehreren Nodes gleichzeitig zu starten und mit diesen einzelne Frames parallel auszuführen, wobei der Algorithmus an sich dabei unverändert bleibt. 
Am Ende werden dann die Ergebnisse der einzelnen Nodes wieder zusammengeführt.

Vorteil hierbei ist eine relativ einfache Implementierung. 

Nachteilig hingegen ist die geringe Flexibilität und das Vernachlässigen des verbleibenden Potentials der Nodes. Des weiteren entsteht ein Overhead durch den Funktionsaufruf bei jedem einzelnen Frame. 

%TODO: pro and contra

\subsection{Jobserver für rechenaufwendigen Funktionen}
\label{subsec:jobserver}

Eine weitere Möglichkeit besteht in der Einführung eines Jobservers, der eine vorgefertigte Menge an rechenaufwendigen Funktionen an mehrere Nodes verteilt und das Ergebnis anschließend zurückgibt. 

%TODO: pro and contra

\subsection{Primitive Aufteilung auf mehrere Nodes mit Jobservern}
 - Frame Pakete erstellen und an einzelne Nodes verschicken, wo diese dann seriell ausgeführt werden und rechenaufwendige Funktionen an einen Jobserver übergeben

%TODO: pro and contra

\subsection{Paketweises Abarbeiten der Daten}
 - Frame Pakete erstellen und auf Nodes verteilen, wo diese dann schrittweise parallel verarbeitet werden und das Paket als ein ganzes Rechenaufwendige Funktionen erreicht. Diese werden dann gleichzeitig parallel verarbeitet. 

%TODO: pro and contra

\section{Implementierung}

\subsection{Implementierte Lösungsstrategie}

\subsection{Softwareimplementierung}
\fi