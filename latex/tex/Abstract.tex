\abstractde{Ziel dieser Arbeit war es, einen bereits von Elena-Ruxandra Cojocaru und Sébastien Bérujon in Python implementierten Wellenfrontrekonstruktionsalgorithmus zu beschleunigen. Dieser berechnet aus zwei Bildern eines zu untersuchenden Objekts pixelweise die Fronten der elektromagnetischen Welle eines Röntgenlasers. Die Bilder werden dabei von zwei in einem fest zueinander Abstand stehenden, hochempfindlichen Röntgen-CCD-Sensoren aufgenommen. Die Kenntnis der Wellenfront ist die Grundlage zur Phasenrekonstruktion bei Röntgenstreubildern aus denen die Struktur von Proben abgeleitet werden kann, die mit Hilfe des Röntgenlasers untersucht werden. Auf Basis von Performance-Analysen der Python-Implementierung wurden Optimierungen und Parallelisierungsmöglichkeiten für die kritischen Programmabschnitte ermittelt, implementiert und evaluiert. Die zu verarbeitenden Bildpaare wurden hierfür auf mehrere Rechenkerngruppen verteilt, auf welchen die Verarbeitung dieser ebenfalls parallelisiert wurde. Zusätzlich zur Parallelisierung wurden verschiedene Methoden der Übersetzung des Python-Codes und die Nutzung von bereits optimierten Funktionen betrachtet. Die hier präsentierte Lösung bietet einen Beschleunigungsfaktor zwischen 1,3 und vier gegenüber der Referenzimplementierung ohne die Zuhilfenahme weiterer Rechenkerne. Unter Nutzung von 120 Kernen wurden Beschleunigungsfaktoren von bis zu 133 gegenüber der Referenzimplementierung auf einem einzelnen Rechenkern erreicht. Darüber hinaus lässt sich das Problem mithilfe der neuen Lösung unabhängig von der Anzahl der Bildpaare über mehrere Rechner verteilen. Die Referenzdaten hierfür wurden an der Beamline BM05 der European Synchrotron Radiation Facility aufgenommen. }

\abstracten{\correctme{TODO}}