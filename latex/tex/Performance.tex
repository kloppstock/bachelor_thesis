\chapter{Performance-Messungen der parallelen Implementation}

\section{Evaluierung der Optimierungen}

\subsection{Parallelisierung}

\begin{correctmore}
	- Speedup: 2-5x (geschätzt; ohne mehr nodes)
	- linear mit Anzahl der Nodes bis Anzahl der Bildpaare
\end{correctmore}

\subsection{Optimierung von Python Engpässen}

\subsubsection{Intrinsics}

\begin{correctmore}
	TODO
\end{correctmore}

\subsubsection{Kompilieren}

\begin{correctmore}
TODO
\end{correctmore}

\section{Einfluss der Parameter}

\begin{correctmore}
	- Algorithmus in verschiedenen Konfigurationen benchmarken (Framepaket Größe variieren, ...)
	- Performance Sweet-Spot finden
\end{correctmore}

\section{Skalierung}

\subsection{Skalierungsfaktor}

\correctme{ - auf Skalierungsfaktor eingehen und diesen in Bezug zu Parametern und verwendeter Hardware setzen}

\subsection{Sättigung}

\begin{correctmore}
	- Sättigungspunkt -grund beschreiben
	- Sättigungsgrund:
	-> Framepaketgröße: Rechenaufwand >> Kopieraufwand -> keine weitere Optimierung des Kopieraufwandes möglich
	-> Anzahl der Nodes: Kopieraufwand übersteigt Rechenaufwand bzw. nähert sich an
	-> nur begrenzte Menge an Datenverfügbar
\end{correctmore}