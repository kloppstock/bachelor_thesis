\chapter{Einleitung}

Die Programmiersprache Python erfreut sich heutzutage besonders bei Physikern hoher Popularität. Sie ist einfach zu bedienen und bietet Zugriff auf eine große Ansammlung physikalischer Hilfsbibliotheken, was unter anderem auch die Portierung von MATLAB-Code vereinfacht. Der hier zu parallelisierende Wellenfrontrekonstruktionsalgorithmus wurde auf diese Weise von Elena-Ruxandra Cojocaru auf Grundlage des von Sébastion Bérujon entwickelten MATLAB-Codes an der \gls{ESRF} zu Python portiert. Die Aufgabe dieses Algorithmus ist es, aus zwei Bildern eines zu untersuchenden Objektes, die Fronten der elektromagnetischen Welle eines Röntgenlasers zu rekonstruieren. Die Bilder werden hierbei von zwei hochempfindlichen Röntgen-CCD-Sensoren aufgenommen, welche in einem festen Abstand zueinander positioniert sind. Diese haben jeweils eine Auflösung von 2048 mal 2048 Pixeln und können bis zu zehn Bilder in einer Sekunde aufnehmen. 

Der vorgegebene Python-Code benötigt jedoch pro Bildpaar mehr als 500 Sekunden auf einem Kern eines Intel\textregistered \mbox{Xeon\textregistered} E5-2680 v3 \glspl{CPU}. Selbst unter der Nutzung von 24 dieser Rechenkerne unterschreitet die Laufzeit pro Bildpaar die 100-Sekunden-Grenze nicht.

Ziel dieser Arbeit ist es daher, die Laufzeit des Algorithmus näher an eine mögliche Echtzeitausführung des Programmes zu bringen. Dabei soll die hier präsentierte Lösung eine möglichst gute Skalierung einerseits mit den Bildpaaren und andererseits mit den \gls{CPU}-Kernen liefern. Auch eine einfache Erweiterbarkeit des Python-Codes ist erwünscht. 

Um diese Kriterien erfüllen zu können, wurde daher zuerst der von \citeauthor{Coj17} entwickelte Code bezüglich seiner Performance evaluiert. Diese Kenntnisse wurden anschließend für die Parallelisierung und weitere Optimierung genutzt. Hierbei wurden verschiedene Optimierungsmöglichkeiten betrachtet, welche sich vor allem mit der Nutzung bereits optimierter Bibliotheken und dem Übersetzen des Programmes in Maschinencode befassen. Zum Schluss wurden die Optimierungen bezüglich des erreichten Beschleunigungsfaktors gegenüber der Referenzimplementierung evaluiert.

Die hier verwendeten Datensätze wurden von Elena-Ruxandra Cojocaru, Sébastion Bérujon und Eric Ziegler am 24. September 2017 und von Thomas Roth, Raymond Barett, Sébastion Bérujon und Rafael Celeste am zehnten April 2017 an der Beamline BM05 der \gls{ESRF} aufgenommen. Diese Arbeit wurde als Teil des \gls{EUCALL}-Projekts am \gls{HZDR} geschrieben, welches von der Europäischen Union finanziert wird. Konkret ist diese Arbeit ein übergreifendes Projekt über die Arbeitspakete WP5 (\gls{UFDAC}) und WP7 (\gls{PUCCA}). 