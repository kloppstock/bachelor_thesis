\chapter{Einleitung}

\begin{correctmore}
	In dieser Arbeit wird ein bereits in Python implementierter Wellenfrontrekonstruktionsalgorithmus für Manycore-Prozessoren beschrieben.
	Die Wellenfront wird aus zwei Bildern ein und desselben Targets rekonstruiert, die von zwei CCD-Sensoren aus unterschiedlicher Entfernung aufgenommen werden. Aus den Unterschieden der Bilder lässt sich der Verlauf der Wellenfornten herleiten, welcher wiederum Aufschluss über die dreidimensionale Struktur des Targets gibt. 
	
	In seiner momentanen Implementierung benötigt der Algorithmus 25 Sekunden pro Bildpaar \textbf{nachprüfen}. Da die CCD-Sensoren mit bis zu 25 Bildern pro Sekunde erhebliche Datenmengen erzeugen, ist es wünschenswert diese Daten in-time zu verarbeiten, sodass eine Speicherung der Daten entfällt. 
	
	Soll-Kriterien, die hierbei an die parallele Lösung gestellt werden, sind zuallererst die Korrektheit der parallelen Implementierung und ein erheblicher Speedup \textbf{von wie viel?}, wobei eine echtzeitfähige Implementierung angestrebt wird.
\end{correctmore}	

\iffalse
- Anwendungsbeschriebung:
-> Wellenfrontrekonstruktionsalgorithmus liegt in Python implementiert vor
-> Rekonstruktion der Wellenfront aus Bildern 2er CCD Kameras in verschiedenem Abstand
- Motivation
-> momentaner Wellenfrontrekonstruktionsalgorithmus benötigt 25s/Frame
-> Kamera liefert 10 Frames/s
-> Daten müssen zwischengespeichert werden
- Einführen der Soll-Kriterien:
-> korrekte Rekonstruktion der Wellenfronten
-> deutliche Beschnleunigung der momentanen Implementierung (5x+)
\fi