\copyrighterklaerung{
Die hier präsentierte Lösung basiert auf dem von Elena-Ruxandra Cojocaru implementierten Python Code, welcher wiederum auf Gundlage des MATLAB-Codes von Sébastien Bérujon erstellt wurde. Der Code der ursprünglichen Implementierung lässt sich auf dem GitHub-Repository \textit{Wavefront-Sensor} unter der General Public License finden\footnote{\url{https://github.com/ComputationalRadiationPhysics/Wavefront-Sensor/blob/master/LICENSE}} \cite{Coj17}.

Ebenso wurden, mit Erlaubniss der Ersteller, an der Beamline BM05 der \gls{ESRF} gemessene Daten zum Testen und Evaluieren verwendet. Dies waren die Datensätze \textit{Experiment 6}, welche am 24. September 2017 von Ruxandra Cojocaru, Sébastien Bérujon und Eric Ziegler aufgenommen wurde und die \textit{Lenses}-Datensätze, welche am zehnten April 2017 von Thomas Rothand, Raymond Barett, Sébastien Bérujon and Rafael Celeste erfasst wurden. 

Zur Implementierung des Wellenfrontrekonstruktionsalgorithmus wurden viele Open Source-Bibliotheken verwendet. Folgende Bibliotheken wurden hier besonders intensiv genutzt:
\begin{itemize}
	\item Cython (Apache Lizenz\footnote{\url{https://www.apache.org/licenses/LICENSE-2.0}})
	\item EdfFile.py (GPL Version 2\footnote{\label{gplv2}\url{https://www.gnu.org/licenses/gpl-2.0.html}})
	\item joblib (BSD\footnote{\label{bsd}\url{https://opensource.org/licenses/BSD-2-Clause}})
	\item matplotlib (matplotlib Lizenz\footnote{\url{https://matplotlib.org/users/license.html}})
	\item mpi4py (BSD\cref{bsd})
	\item numba (BSD\cref{bsd})
	\item NumPy (BSD 2.0\footnote{\label{bsd2}\url{https://opensource.org/licenses/BSD-3-Clause}})
	\item OpenCV (BSD\cref{bsd})
	\item Pillow (Standard PIL Lizenz\footnote{\url{http://www.pythonware.com/products/pil/license.htm}})
	\item Python (Python Software Foundation License\footnote{\url{https://www.python.org/download/releases/2.7/license/}})
	\item SciPy (SciPy Lizenz\footnote{\url{https://scipy.org/scipylib/license.html}})
\end{itemize}
}